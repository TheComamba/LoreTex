\documentclass[openany]{book}

\usepackage{luacode}
\usepackage{nameref}

\newcommand{\LoreTex}{Lore\TeX}
\newcommand{\LoreTexSort}[1]{}

\newcommand{\loadLuacode}[1]{
    \directlua{
        dofile([[#1]] .. "/state.lua")
        dofile([[#1]] .. "/common.lua")

        dofile([[#1]] .. "/comparer.lua")
        dofile([[#1]] .. "/debug.lua")
        dofile([[#1]] .. "/dates-and-years.lua")
        dofile([[#1]] .. "/dictionary.lua")
        dofile([[#1]] .. "/errors.lua")
        dofile([[#1]] .. "/scoped-variables.lua")
        dofile([[#1]] .. "/translation.lua")
        dofile([[#1]] .. "/units.lua")

        dofile([[#1]] .. "/entities/affiliations.lua")
        dofile([[#1]] .. "/entities/api.lua")
        dofile([[#1]] .. "/entities/characters.lua")
        dofile([[#1]] .. "/entities/common.lua")
        dofile([[#1]] .. "/entities/descriptors.lua")
        dofile([[#1]] .. "/entities/getEntities.lua")
        dofile([[#1]] .. "/entities/height.lua")
        dofile([[#1]] .. "/entities/history-processing.lua")
        dofile([[#1]] .. "/entities/history-api.lua")
        dofile([[#1]] .. "/entities/nameMarkers.lua")
        dofile([[#1]] .. "/entities/names.lua")
        dofile([[#1]] .. "/entities/print-descriptors.lua")
        dofile([[#1]] .. "/entities/print.lua")
        dofile([[#1]] .. "/entities/processing.lua")
        dofile([[#1]] .. "/entities/protected-descriptors-api.lua")
        dofile([[#1]] .. "/entities/protected-descriptors-get-set.lua")
        dofile([[#1]] .. "/entities/refs.lua")
        dofile([[#1]] .. "/entities/species.lua")
        dofile([[#1]] .. "/entities/subentity.lua")
        dofile([[#1]] .. "/entities/types.lua")
        dofile([[#1]] .. "/entities/visibility.lua")

        dofile([[#1]] .. "/ffi/convert.lua")
        dofile([[#1]] .. "/ffi/lorecore.lua")
        dofile([[#1]] .. "/ffi/reading-database.lua")
        dofile([[#1]] .. "/ffi/writing-database.lua")

        ResetState()
        RelativePath = [[#1]]
    }

    \newcommand{\setCurrentYear}[1]{\directlua{CurrentYear = #1}}
\newcommand{\setCurrentDay}[1]{\directlua{CurrentDay = #1}}
\newcommand{\setDaysPerYear}[1]{\directlua{DaysPerYear = #1}}

\newcommand{\addDayFmt}[1]{\directlua{TexApi.addDayFmt("#1")}}
\newcommand{\addYearFmt}[1]{\directlua{TexApi.addYearFmt("#1")}}
\newcommand{\printDay}[1]{\directlua{tex.print(DayString(#1))}}
\newcommand{\printYear}[1]{\directlua{tex.print(YearString(#1))}}

\newcommand{\setYearAbbreviation}[1]{\directlua{TexApi.setYearAbbreviation([[#1]])}}
\newcommand{\setYearOffset}[1]{\directlua{TexApi.setYearOffset(#1)}}
\newcommand{\addMonth}[2]{\directlua{TexApi.addMonth{ month = [[#1]], firstDay = #2 }}}
    \newcommand{\listAllRefs}{\directlua{Debug.listAllRefs()}}
\newcommand{\makeFirstEntitiesPrimary}[1]{\directlua{Debug.makeFirstEntitiesPrimary(#1)}}
\newcommand{\printAllTypes}{\directlua{Debug.printAllTypes()}}
\newcommand{\printDescriptorsImmediately}{\directlua{Debug.printDescriptorsImmediately()}}
\newcommand{\debugAutomatedChapters}{\directlua{Debug.debugAutomatedChapters()}}

    \begin{luacode*}
    dofile(RelativePath .. "/dictionary.lua")
\end{luacode*}
    \newcommand{\PrintRpgTexErrors}{\directlua{tex.print(PrintErrors())}}
\newcommand{\throwOnError}{\directlua{IsThrowOnError = true}}
    \newcommand{\setEntitiesPath}[1]{ \directlua{ SetScopedVariable( "FilepathToEntities", [[#1]] ) } }

\newcommand{\loadChild}[2][]{
    \directlua{ LoadChildFile( [[#1]], [[#2]]) }
    \directlua{ PopScopedVariables() }
}

\newcommand{\setDefaultLocation}[1]{ \directlua{ SetScopedVariable( "DefaultLocation", "#1" ) } }
    \newcommand{\resetLoreTex}{\directlua{ResetState()}}
    \newcommand{\SelectLanguage}[1]{\directlua{SelectLanguage("#1")}}

\SelectLanguage{english}
    \newcommand{\ft}[1]{\directlua{PrintFtToM(#1)}}
\newcommand{\ftsquared}[1]{\directlua{PrintFtToMSquared(#1)}}
\newcommand{\ftcubed}[1]{\directlua{PrintFtCubedToLiter(#1)}}
\newcommand{\metre}[1]{\directlua{PrintMToFt(#1)}}
\newcommand{\inch}[1]{\directlua{PrintInToCM(#1)}}
\newcommand{\mile}[1]{\directlua{PrintMiToKm(#1)}}
\newcommand{\lb}[1]{\directlua{PrintLbToKg(#1)}}
\newcommand{\gallon}[1]{\directlua{PrintGallonToLiter(#1)}}
\newcommand{\ounce}[1]{\directlua{PrintOunceToLiter(#1)}}
\newcommand{\quart}[1]{\directlua{PrintQuartToLiter(#1)}}
\newcommand{\fahrenheit}[1]{\directlua{PrintFahrenheitToCelsius(#1)}}


    \documentclass[openany]{book}

\usepackage{luacode}
\usepackage{nameref}

\newcommand{\LoreTex}{Lore\TeX}
\newcommand{\LoreTexSort}[1]{}

\newcommand{\loadLuacode}[1]{
    \directlua{
        dofile([[#1]] .. "/state.lua")
        dofile([[#1]] .. "/common.lua")

        dofile([[#1]] .. "/comparer.lua")
        dofile([[#1]] .. "/debug.lua")
        dofile([[#1]] .. "/dates-and-years.lua")
        dofile([[#1]] .. "/dictionary.lua")
        dofile([[#1]] .. "/errors.lua")
        dofile([[#1]] .. "/scoped-variables.lua")
        dofile([[#1]] .. "/translation.lua")
        dofile([[#1]] .. "/units.lua")

        dofile([[#1]] .. "/entities/affiliations.lua")
        dofile([[#1]] .. "/entities/api.lua")
        dofile([[#1]] .. "/entities/characters.lua")
        dofile([[#1]] .. "/entities/common.lua")
        dofile([[#1]] .. "/entities/descriptors.lua")
        dofile([[#1]] .. "/entities/getEntities.lua")
        dofile([[#1]] .. "/entities/height.lua")
        dofile([[#1]] .. "/entities/history-processing.lua")
        dofile([[#1]] .. "/entities/history-api.lua")
        dofile([[#1]] .. "/entities/nameMarkers.lua")
        dofile([[#1]] .. "/entities/names.lua")
        dofile([[#1]] .. "/entities/print-descriptors.lua")
        dofile([[#1]] .. "/entities/print.lua")
        dofile([[#1]] .. "/entities/processing.lua")
        dofile([[#1]] .. "/entities/protected-descriptors-api.lua")
        dofile([[#1]] .. "/entities/protected-descriptors-get-set.lua")
        dofile([[#1]] .. "/entities/refs.lua")
        dofile([[#1]] .. "/entities/species.lua")
        dofile([[#1]] .. "/entities/subentity.lua")
        dofile([[#1]] .. "/entities/types.lua")
        dofile([[#1]] .. "/entities/visibility.lua")

        dofile([[#1]] .. "/ffi/convert.lua")
        dofile([[#1]] .. "/ffi/lorecore.lua")
        dofile([[#1]] .. "/ffi/reading-database.lua")
        dofile([[#1]] .. "/ffi/writing-database.lua")

        ResetState()
        RelativePath = [[#1]]
    }

    \input{#1/dates-and-years.tex}
    \input{#1/debug.tex}
    \input{#1/dictionary.tex}
    \input{#1/errors.tex}
    \input{#1/scoped-variables.tex}
    \input{#1/state.tex}
    \input{#1/translation.tex}
    \input{#1/units.tex}

    \input{#1/entities/history.tex}
    \input{#1/entities/protected-descriptors.tex}
    \input{#1/entities/common.tex}
    \input{#1/entities/refs.tex}
    \input{#1/entities/visibility.tex}

    \input{#1/ffi/reading-database.tex}
    \input{#1/ffi/writing-database.tex}
}

\loadLuacode{../../src/}
\throwOnError

\begin{document}

The problem of history shared between various entities is what spawned this project. The most noteworthy events in the history of e.g. a character are usually those that affect others as well. This means that their character entry should probably display the same event. In the past, I used to copy events to all entities it concerned, but this was tedious, error-prone, and hard to maintain. \LoreTex{} fixes that.

The command to add history to an entity is:
\begin{verbatim}
\addHistory[day]{year}{event}
\end{verbatim}
Note that giving a day is optional.

\newEntity{NPCs}{flora}{}{Flora}

\addHistory{1823}{
	\nameref{flora} begins to speak.
}

Any event that references another entity is added to that entity as well.

\addHistory{1830}{
	\nameref{flora} befriends \nameref{paul}.
}

If an event does not mention another entity, but nonetheless concerns it, you can use the concerns command to signal exactly that. The opposite command, notconcerns, also exists, to signal that an event should not be added to some entity even though it is mentioned.

\addHistory{1832}{
	\nameref{flora} moves to \nameref{rome}.
	\concerns{paul}
	\notconcerns{rome}
}

Two more important commands to signal the birth and death (or, more generally, the coming in and out of existence) of an entity also exists: birthof and deathof.

\addHistory{1820}{
	The twins \nameref{flora} and \nameref{ramona} are born.
	\birthof{flora}
	\birthof{ramona}
}

\addHistory{1878}{
	\nameref{flora} dies.
	\deathof{flora}
}

To put histories into context, you can specify the current year and day of the current year (both default to 0 otherwise), as well as the number of days your year has:
\begin{verbatim}
\setCurrentYear{year}
\setCurrentDay{day}
\setDaysPerYear{daysPerYear}
\end{verbatim}
\setCurrentYear{1888}
\setCurrentDay{166}
\setDaysPerYear{365}
Note, for example, that \LoreTex{} calculates the current age of characters whose birthyear was given.


\newEntity{NPCs}{paul}{}{Paul}

\addHistory[32]{1888}{
	\nameref{paul} finds out about \nameref{flora}'s demise.
	\notconcerns{flora}
}

\newEntity{NPCs}{ramona}{}{Ramona}
\newEntity{places}{rome}{}{Rome}


\makeEntityPrimary{flora}
\makeEntityPrimary{paul}
\makeEntityPrimary{ramona}
\makeEntityPrimary{rome}

\automatedChapters{}

\end{document}
    \newcommand{\setAgeExponent}[1]{\directlua{TexApi.setAgeExponent(#1)}}
\newcommand{\setAgeFactor}[1]{\directlua{TexApi.setAgeFactor(#1)}}
\newcommand{\setAgeModifierMixing}[2]{\directlua{TexApi.setAgeModifierMixing("#1","#2")}}
\newcommand{\addParent}[2][]{\directlua{TexApi.addParent{ parentLabel = "#2", relationship = [[\detokenize{#1}]] }}}
\newcommand{\setHeight}[1]{\directlua{TexApi.setHeight(#1)}}
\newcommand{\setLocation}[1]{\directlua{TexApi.setLocation("#1")}}
\newcommand{\setSecret}{\directlua{TexApi.setSecret()}}
\newcommand{\setSpecies}[1]{\directlua{TexApi.setSpecies([[\detokenize{#1}]])}}
    \newcommand{\newEntity}[4]{\directlua{TexApi.newEntity{ category = "#1", label = "#2", shortname = [[\detokenize{#3}]], name = [[\detokenize{#4}]] }}}
\newcommand{\newNPC}[3]{\directlua{TexApi.newCharacter{ label = "#1", shortname = [[\detokenize{#2}]], name = [[\detokenize{#3}]] }}}
\newcommand{\declarePC}[1]{\directlua{TexApi.declarePC("#1")}}
\newcommand{\addDescriptor}[3][]{\directlua{TexApi.setDescriptor{ descriptor = [[\detokenize{#2}]], description = [[\detokenize{#3}]], subdescriptor = [[\detokenize{#1}]] }}}
\newcommand{\automatedChapters}{\directlua{tex.print(TexApi.automatedChapters())}}
\newcommand{\newpageAfterEntity}{\directlua{NewpageAfterEntity = true}}


    \begin{luacode*}
    dofile(RelativePath .. "/refs.lua")
\end{luacode*}

\newcommand{\makeEntityPrimary}[1]{\directlua{AddRef("#1", PrimaryRefs)}}
\newcommand{\makeAllEntitiesOfTypePrimary}[1]{\directlua{MakePrimaryIf(Bind(IsType, #1))}}
\newcommand{\mention}[1]{\directlua{AddRef("#1", SecondaryRefs)}}
\newcommand{\reference}[1]{\makeEntityPrimary{#1}\nameref{#1}}
\newcommand{\itref}[1]{\textit{\nameref{#1}}}
\newcommand{\listAllRefs}{\directlua{ListAllRefs()}}
    \newcommand{\reveal}[1]{\directlua{TexApi.reveal("#1")}}
\newcommand{\showSecrets}{\directlua{TexApi.showSecrets()}}
\newcommand{\showFuture}{\directlua{TexApi.showFuture()}}


    \newcommand{\readLoreFromDatabase}[1]{\directlua{TexApi.readLoreFromDatabase([[#1]])}}
    \newcommand{\writeLoreToDatabase}[1]{\directlua{TexApi.writeLoreToDatabase([[#1]])}}
}

\loadLuacode{../../src/}
\throwOnError

\begin{document}

If an imaginary world grows large, it becomes ever more important to sort entities. The topmost sorting layer is that of entity type - items and species will appear in separate chapters. Below that, the entities are sorted by location. A location can be any other entity of any type. It usually makes most sense for a location to be of type places, but \LoreTex{} doesn't care. One plausible counterexample is a ship, which serves as the location of its crew. A location is set using the setLocation command with the label of the location after declaring a new entity.

A location can itself have a location, allowing to make the sorting exactly as fine grained as desired. Locations do not have to be referenced in the main text to work. Any entities that have another entity as a location are listed in the entry of that location, sorted by their type.

\newEntity{places}{outer-rim}{}{Outer Rim}

\newEntity{places}{tattooine}{}{Tattooine}
\setLocation{outer-rim}

\addDescriptor{Description}{A desert planet.}

\newEntity{places}{mos-eisley}{}{Mos Eisley}
\setLocation{tattooine}

\newEntity{species}{tusken}{}{Tusken}
\setLocation{tattooine}

\newEntity{species}{jawa}{}{Jawa}
\setLocation{tattooine}

\makeEntityPrimary{tattooine}
\makeEntityPrimary{mos-eisley}

Associations are a similar concept, but they do not influence how an entity is sorted. They are set using the addParent command with the label of the associated entity. Both entities will acknowledge the connection if they are printed out.

The addParent command can be given a role in the association as an optional argument. Otherwise they will simply be called member. An entity can have an arbitrary number of associations.

\newEntity{other}{galactic-empire}{}{Galactic Empire}

\newEntity{other}{star-destroyer}{}{Star Destroyer}
\addParent{galactic-empire}

\newEntity{other}{death-star}{}{Death Star}
\addParent[Flawless Flagship]{galactic-empire}

\newEntity{other}{death-star-2}{}{Death Star 2}
\addParent[Unfinished Flagship]{galactic-empire}
\addParent[Successor]{death-star}

\makeEntityPrimary{galactic-empire}
\makeEntityPrimary{star-destroyer}
\makeEntityPrimary{death-star}




\automatedChapters{}

\end{document}